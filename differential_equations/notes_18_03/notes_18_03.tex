\documentclass{article}

\usepackage{technical_notes}

\makeindex

\begin{document}

\title{Notes for MIT 18.03: Differential Equations}
\author{Andrew Winnicki}
\date{Spring 2025}
\maketitle

\section{Recitation 1 \& Lecture 1: Direction Fields, Existence \& Uniqueness of Solutions}

\subsection{Readings}

Edwards \& Penney, Chapters 1.1, 1.2, 1.3, and 1.4.

\subsection{Lecture \& Reading Notes}

\subsubsection{Differential Equations \& Separability}

In the the first part of the course, we will only study ODEs involving the first derivative:

\begin{equation}
  \label{eq:firstorderode}
  y' = F(x, y)
\end{equation}

Example:
\begin{equation}
  y' = 2x,
\end{equation}

which has solution $x^2 + C$. Such a solution is called the \textbf{general solution}\index{general solution}.

Another example \textit{(memorize this!)}:

\begin{equation}
  \label{eq:naturalgrowth}
  y' = kx
\end{equation}

With general solution $y = Ce^{kx}$. This will be a central example in the course; in fact, a reasonable defintion of the function $e^x$ is that it is the solution to the differential equation $y' = y$ such that $y(1) = 0$.

Separation of variables\index{separation of variables} will also be important. The protocol is:
\begin{enumerate}
\item Pull all variables to either side.
\item Integrate.
\item Amalgamate all constants and solve (if possible) for $y$ in terms of $x$.
\end{enumerate}

Most real-life equations aren't solvable, though. So, we often resort to graphical/numerical methods.

\subsubsection{Graphical Methods}

The differential equation $y' = F(x, y)$ defines a slope at every point in the plane. This is a \textbf{direction field}\index{direction field} or \textbf{slope field}\index{slope field}.

A \textbf{solution}\index{solution} of a differential equation is a function whose graph has the given slope at every point it goes through.

To specify a particular solution of an ODE you must give an \textbf{initial condition}\index{initial condition}: when $x$ takes on a certain value, $y$ takes on a specified value.

\begin{Theorem}{Existence and Uniqueness Theorem for ODEs}{testtheorem}
   Suppose that both the function $f(x, y)$ and $D_y f(x, y)$ are continuous on some rectangle $R$ in the $xy$-plane containing the point $(a, b)$ in its interior. Then for some open interval $I$ containing the point $a$, the IVP
  \begin{equation}
    \label{eq:ivp_existence_uniqueness_theorem}
    \frac{dy}{dx} = f(x, y), \quad y(a) = b
  \end{equation}
  has one and only one solution defined on the interval $I$.
\end{Theorem}

Direction fields are used to visualize the qualitative behavior of solutions to differential equations, and this is what we often want to know.

\section{Lecture 2: Numerical Methods}

\subsection{Readings}

Edwards \& Penney 6.1, 6.2.

\subsection{Lecture \& Reading Notes}

The study of differential equations has three parts.

\begin{enumerate}
\item Analytic, exact, symbolic methods.
\item Quantitative methods (direction fields, isoclines, etc).
\item Numerical methods.
\end{enumerate}

We focus on numerical methods today.

\printindex

\end{document}
